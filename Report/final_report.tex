%%%%%%%%%%%%%%%%%%%%%%%%%%%%%%%%%%%%%%%%%
% University Assignment Title Page 
% LaTeX Template
% Version 1.0 (27/12/12)
%
% This template has been downloaded from:
% http://www.LaTeXTemplates.com
%
% Original author:
% WikiBooks (http://en.wikibooks.org/wiki/LaTeX/Title_Creation)
%
% License:
% CC BY-NC-SA 3.0 (http://creativecommons.org/licenses/by-nc-sa/3.0/)
% 
% Instructions for using this template:
% This title page is capable of being compiled as is. This is not useful for 
% including it in another document. To do this, you have two options: 
%
% 1) Copy/paste everything between \begin{document} and \end{document} 
% starting at \begin{titlepage} and paste this into another LaTeX file where you 
% want your title page.
% OR
% 2) Remove everything outside the \begin{titlepage} and \end{titlepage} and 
% move this file to the same directory as the LaTeX file you wish to add it to. 
% Then add \input{./title_page_1.tex} to your LaTeX file where you want your
% title page.
%
%%%%%%%%%%%%%%%%%%%%%%%%%%%%%%%%%%%%%%%%%
%\title{Title page with logo}
%----------------------------------------------------------------------------------------
%   PACKAGES AND OTHER DOCUMENT CONFIGURATIONS
%----------------------------------------------------------------------------------------

\documentclass[12pt]{article}
\usepackage[english]{babel}
\usepackage[utf8x]{inputenc}
\usepackage{amsmath}
\usepackage{graphicx}
\usepackage[colorinlistoftodos]{todonotes}
\usepackage{hyperref}
\usepackage{titlesec}

\begin{document}

\begin{titlepage}

\newcommand{\HRule}{\rule{\linewidth}{0.5mm}} % Defines a new command for the horizontal lines, change thickness here

\center % Center everything on the page
 
%----------------------------------------------------------------------------------------
%   HEADING SECTIONS
%----------------------------------------------------------------------------------------

\textsc{\LARGE University California Riverside}\\[1.5cm] % Name of your university/college
\textsc{\Large CS179J Senior Design Project}\\[0.5cm] % Major heading such as course name
\textsc{\large Spring 2017}\\[0.5cm] % Minor heading such as course title

%----------------------------------------------------------------------------------------
%   TITLE SECTION
%----------------------------------------------------------------------------------------

\HRule \\[0.4cm]
{ \huge \bfseries Smart Fridge}\\[0.4cm] % Title of your document
\HRule \\[1.5cm]
 
%----------------------------------------------------------------------------------------
%   AUTHOR SECTION
%----------------------------------------------------------------------------------------

\begin{minipage}{0.4\textwidth}
\begin{flushleft} \large
\emph{Author:}\\
Stanley \textsc{Cohen} % Your name
\end{flushleft}
\end{minipage}
~
\begin{minipage}{0.4\textwidth}
\begin{flushright} \large
\emph{Professor:} \\
Frank \textsc{Vahid} % Supervisor's Name
\end{flushright}
\end{minipage}\\[2cm]

% If you don't want a supervisor, uncomment the two lines below and remove the section above
%\Large \emph{Author:}\\
%John \textsc{Smith}\\[3cm] % Your name

%----------------------------------------------------------------------------------------
%   DATE SECTION
%----------------------------------------------------------------------------------------

{\large \today}\\[2cm] % Date, change the \today to a set date if you want to be precise

%----------------------------------------------------------------------------------------
%   LOGO SECTION
%----------------------------------------------------------------------------------------
\includegraphics[scale=.5]{logo.png}\\[1cm] % Include a department/university logo - this will require the graphicx package
%----------------------------------------------------------------------------------------

\vfill % Fill the rest of the page with whitespace

\end{titlepage}


\begin{abstract}
The Smart Fridge is a fridge designed to help users keep track of their fridge contents and buying habits. 
    It consists of three main systems; a system at the fridge to monitor insertions and removals, 
    a system in the cloud to store all data and a smartphone system to communicate with the user.
\end{abstract}

\section{User Guide}


Once the fridge is installed, simply plug the Raspberry Pi on to boot up the system. 
    A green LED will light when the database has been properly fetched and the system is ready for inputs.
    At this point, simply scan your items with the barcode scanner before you put them into or remove them from the fridge.
    Your insertions/removals will be automatically communicated to the database that is always waiting for more information.\\

All of your firdge information can be seen in the Android phone app. Once the app has been installed, simply navigate between
    the \textit{Dashboard}, \textit{Fridge} and \textit{History} tabs to get more information on the contents of your fridge
    and your past purchases.
\section{Technologies and Components}

\begin{itemize}
	\item Raspberry Pi
	\item Arduino Uno
    \item IR LED/Reciever
	\item Firebase NoSQL Database
    \item Firebase Backend Javascript Cloud Functions
	\item Android App (built with Java/XML)
\end{itemize}

\section{Demo Video}

Demo video can be found \href{https://youtu.be/m8oFfEKpmIg}{here}.

\section{Source Code}
\label{sec:code}

\subsection{Raspberry Pi}

\subsubsection{main.py}

This is the main script for the fridge. It is in the \textit{bashrc} file so that when the Pi is booted into terminal mode, 
    it will be run automatically (no monitor needed!). The script fetches the current information from the database and listens for
    barcode scans from the scanner and IR beam breaks from the Arduino.\\

If it senses a beam break and then a scan, it will write to the database that the food has been taken out of the fridge. 
    If it senses a scan and then a beam break, it will write to the database that the food has been put in.

\subsection{Arduino}

\subsubsection{IR\_sensor.ino}

Listens for an IR beam break. On beam break, sends the signal to the Raspberry Pi.

Also looks for a button press. On button press, requests fridge data from Pi to send to Dylan via Software Serial.

\subsection{Firebase}

\subsubsection{index.js}

Backend JavaScript used to made reads and writes to the database a LOT easier. Also every minute checks to see if there is expired food.

\subsection{Android}

\subsubsection{MainActivity.java}

Main Java file. Attach listeners to the database nodes and handle tab switching along with most everything else in the app.

\subsubsection{Food.java}

Simple Java class to store a food. Mostly just a bunch of datafields. The only complicated thing here is the string to date converter that matches the default Friebase format.

\subsubsection{FoodListAdapter.java}

Custom ListView adapter for displaying lists of food. Takes an array of arrays of foods and ensures they're displayed properly when their views are called. Also added a function, \textit{getUsage} for getting the number of purchases on any given day of the week to make it easy to create the graph.

\end{document}